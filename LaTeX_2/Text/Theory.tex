\section{Theory}

The theory section should contain background theory relevant for the reader in order to understand the rest of your report.\textbf{ Assume that the reader is yourselves at the start of the semester (before you had learnt anything from this course) and include theory accordingly.} Keep in mind that this section should only include theory relevant to the project and the report, it is not meant as a place to show off everything you have ever learnt.



\subsection{Subsections, equations, figures, and tables}\label{subsec:theory_aSubsection}
When you in later sections apply any of the theory you can refer back to this section. To make it easier for the reader to understand which part you are referring back to, it can be a good idea to divide your sections into subsections (e.g. one subsection per topic). This also makes it easier to read.

Use equations, figures and tables to help get your message across. All figures/tables/equations should be referenced in the text, as they are there to help you tell your story. Remember to cite your references \cite{example}. Every table and figure should be referenced in the text, see \autoref{tab:my_label}. You should also make sure to tell the reader what we are supposed to see in the figure, e.g. \textit{We see from the plot that the signal x goes low 1ns after y goes high, which means that spec 3 has been met.}

\begin{table}[h]
    \centering
    \caption{The caption should give a short description of what the reader can see from this table/plot/figure.}
    \begin{tabular}{c|c}
         & Referenced in text? \\
         \hline
        Figure & YES\\
        Table & YES\\
    \end{tabular}
    \label{tab:my_label}
\end{table}

\subsection{Naming conventions}
When writing a report like this, it is important to follow naming conventions. Otherwise you risk confusing the readers, sometimes so much that they can no longer see the cool/smart/interesting thing you did. And since the readers in your case are people who will be giving you a grade, that is extra unfortunate. 

An example of a naming convention is using the same names for signals as those that were given in the project description. If you for some reason are unable to follow a given naming convention, you should make sure to highlight that to the reader so that they understand what is happening.